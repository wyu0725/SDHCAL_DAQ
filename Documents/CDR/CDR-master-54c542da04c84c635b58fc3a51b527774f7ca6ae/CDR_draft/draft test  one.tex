\documentclass{article} \begin{document} 
6.1 General Design Considerations
The CEPC detector magnet system is designed to provide an axial magnetic field over the tracking volume, a diameter 6.8 m over a length of 8.05 m, particle detectors within this volume will measure the trajectories of charged tracks emerging from the collisions. Based on the study of the compensating magnet of CEPC, a 3 T central field of detector superconducting solenoid is more reality to construct the CEPC compensation solenoid which making full cancelation to avoid disturbance to the beam with technologies to be developed in coming years, comparing the 3.5 T solenoidal field proposed in pre-CDR.
This chapter contains all the activities related to the design of field distribution, solenoid coil, specific superconductor, cryogenics, quench protection and power supply, and the yoke.
In this report, for the first time, we explored the possibility of using HTS to build CEPC detector magnet. Benefit from the development of high Tc superconductor in recent years, the advantages by using YBCO winding is higher operating temperature, better stability to resist transient disturbances when operating the magnet.
We also discussed on the choice between iron yoke design and dual solenoid scenario. The baseline design iron yoke consists of barrel yoke and end yoke. It has three main functions: first is shielding the magnetic field; second is providing the install space for the muon detector which sandwiched between the iron plates; in addition, the yoke serves as the main mechanical structure of the CEPC detector. However, the second one which called active shielding design has been widely used in for commercial MRI magnets, this scenario will not use iron yoke, and it has large impact on muon detector design.

\end{document}