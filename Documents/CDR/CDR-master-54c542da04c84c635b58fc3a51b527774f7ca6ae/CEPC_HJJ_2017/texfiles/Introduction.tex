\section{Introduction}\label{sec:introduction}
The discovery of a scalar boson with mass around 125 \GeV at LHC \cite{Higgs_ATLAS, Higgs_CMS} completed the fundamental particles list in the standard model.
This particle, interpreted as the higgs boson, plays a lead role in the electroweak spontaneous symmetry broken(EWSB), 
known as the BEH(Brout-Englert-Higgs) mechanism \cite{BEH,BEH2}.
The higgs meachism guarantees that the $\Wboson$, $\Zboson$, as well as the fermions 
like quarks and charged leptons can be massive in $SU(2)_L \times U(1)_Y$ gauge invariant way.
The mass of the fermions $m_{f_i}$ in standard model and their couplings to the higgs field $h_i$, so called Yukawa coupling, are related proportionally: $m_{f_i} = \dfrac{vh_i}{\sqrt{2}}$, in which $v$ stands for the higgs field vaccume expectation(VEV), evalued around 246 \GeV. 
Therefore measuring the Yukawa coupling between higgs and SM fermions is essential to undertand the origin of the fermions' masses and the detail of EWSB.
The dominant higgs fermionic decay are expected to be $\Hboson\to \bpair$(Br.\percent{57}), $\Hboson\to\cpair$(\percent{4}) and $\Hboson\to\taupair$(\percent{3}). 
In addition, the higgs can also decay to gluon pairs dominantly via a top loop diagram. The large coupling between higgs and top quark lead to considerable 
branch ratio of $\Hboson\to\gpair$(Br. \percent{9}). 
\par
Until now, the LHC is the only place to directly study the higgs experimentally. 
The leading higgs fermionic decay,  $\Hboson\to\bpair$ was studied in both ATLAS and CMS experiment in VH\cite{VH_bb_atlas,VH_bb_cms}, ttH\cite{ttH_bb_cms} and VBF\cite{VBF_bb_atlas,VBF_bb_cms} process, with the LHC Run-I data.
The combination of ATLAS and CMS gives $\bpair$ $\sigma\times Br$ signal strength for $0.70\pm0.29$ in run-I data\cite{higgs_atlas_cms_combine}. The large uncertainty is due to huge QCD or vector boson production with muliti-jets backgrounds, which is inevitable in hadron colliders. 
\par
The Circular Electron Positron Collider(CEPC) \cite{CEPC_preCDR} program is proposed with the goal to better understand the EWSB
 by precisely measuring on these higgs parameters as well as other EW parameters of interest. 
 The CEPC has the advantages in precision measurement:
 \begin{itemize}
 \item Clean backgrounds
 \item Well defined frame of center momentum
 \item High luminosity
 \end{itemize}

% The center of mass enerey is configured as \GeV{250}, well above the $\Zboson\Hboson$ threshold.
% With the nominal luminosity at \ten{2}{34}\cm2s1, \ifb{5000} data can be accumulated in ten years of running. 
% Over one million higgs boson will be produced. Clean background and high statistics make precise measurement possible.
%The outline of higgs background will be discussed in section \ref{sec:CEPC_MC}
%\par
The works presented in this note demostrate the capability of the $\Hboson \to \bpair/\cpair/\gpair$ measurements in CEPC.
The higgs productions associate with charged lepton(electrons or muons) pair, neutrino pair or quark pair are studied.
 In Section II, a brief introduction of CEPC experiment and the MC sample will be presented. In section III the event selection and the analysis strategy will be described. In section IV the results are listed and discussed. Detail information, auxiliary figures, tables and numbers, as well as analysis method in study can be found in appendix.
\par
\clearpage