%%%%%%%%%%%%%%%%%%%%%%%%%%%%%%%%%%%%%%%%%
% NIWeek 2014 Poster by T. Reveyrand
% www.microwave.fr
% http://www.microwave.fr/LaTeX.html
% ---------------------------------------
% 
% Original template created by:
% Brian Amberg (baposter@brian-amberg.de)
%
% This template has been downloaded from:
% http://www.LaTeXTemplates.com
%
% License:
% CC BY-NC-SA 3.0 (http://creativecommons.org/licenses/by-nc-sa/3.0/)
%
%%%%%%%%%%%%%%%%%%%%%%%%%%%%%%%%%%%%%%%%%

%----------------------------------------------------------------------------------------
%	PACKAGES AND OTHER DOCUMENT CONFIGURATIONS
%----------------------------------------------------------------------------------------

\documentclass[a0paper,portrait]{baposter}

\usepackage[font=small,labelfont=bf]{caption} % Required for specifying captions to tables and figures
\usepackage{booktabs} % Horizontal rules in tables
\usepackage{relsize} % Used for making text smaller in some places

\usepackage{amsmath,amsfonts,amssymb,amsthm} % Math packages
\usepackage{eqparbox}

\usepackage{textcomp}

\usepackage{wrapfig}

\usepackage{picinpar}

\usepackage{picins}

\graphicspath{{figures/}} % Directory in which figures are stored

 \definecolor{bordercol}{RGB}{40,40,40} % Border color of content boxes
 \definecolor{headercol1}{RGB}{186,215,230} % Background color for the header in the content boxes (left side)
 \definecolor{headercol2}{RGB}{120,120,120} % Background color for the header in the content boxes (right side)
 \definecolor{headerfontcol}{RGB}{0,0,0} % Text color for the header text in the content boxes
 \definecolor{boxcolor}{RGB}{210,235,250} % Background color for the content in the content boxes


\begin{document}

\background{ % Set the background to an image (background.pdf)
\begin{tikzpicture}[remember picture,overlay]
\draw (current page.north west)+(-2em,2em) node[anchor=north west]
{\includegraphics[height=1.1\textheight]{background}};
\end{tikzpicture}
}

\begin{poster}{
grid=false,
borderColor=bordercol, % Border color of content boxes
headerColorOne=headercol1, % Background color for the header in the content boxes (left side)
headerColorTwo=headercol2, % Background color for the header in the content boxes (right side)
headerFontColor=headerfontcol, % Text color for the header text in the content boxes
boxColorOne=boxcolor, % Background color for the content in the content boxes
headershape=roundedright, % Specify the rounded corner in the content box headers
headerfont=\Large\sf\bf, % Font modifiers for the text in the content box headers
textborder=rectangle,
background=user,
headerborder=open, % Change to closed for a line under the content box headers
boxshade=plain
}
{\includegraphics[width=0.1\textwidth]{figures/CEPCLogoTransparent.png}}
%
%----------------------------------------------------------------------------------------
%	TITLE AND AUTHOR NAME
%----------------------------------------------------------------------------------------
%
{ \bf  \huge {Status of Design and Development of CEPC-DHCAL Prototype Readout Electronics} } % Poster title
{\vspace{0.3em} \smaller Yu Wang$^{1,2}$, Shubin Liu$^{1,2}$, Daojin Hong$^{1,2}$, Jianbei Liu$^{1,2}$, Changqing Feng$^{1,2}$, Yi Zhou$^{1,2}$ \\  % Author names
  
\smaller $^1$\it {University of Science and Technology of China} \\ $^2$\it{State Key Laboratory of Particle Detection and Electronics (USTC)} } % Author email addresses
{\includegraphics[width=0.1\textwidth]{figures/USTCBlueTransparent.png}} % University/lab logo

%----------------------------------------------------------------------------------------
%	INTRODUCTION
%----------------------------------------------------------------------------------------
\headerbox{Introduction}{name=introduction,column=0,row=0, span=3}{
\begin{itemize} 
	\item The goal of this research is to provide a feasible readout scheme for CEPC DHCAL. As the readout channels might be $4\times10^5/m^3$, the readout scheme must be carefully designed.
	\vspace{-0.2cm}
	%\item In PFA-based calorimeters, simulation results suggest that for readout segments as small as $1cm^2$, simple hit counting is already a good energy measurement for hadrons, so called DHCAL. A more general calorimeter with multi-threshold readout (e.g. 3 thresholds) records more detailed hit information and has better energy for jet energy above 40GeV, a so-called Semi-Digital Hadron Calorimeter (SDHCAL)
	%\vspace{-0.2cm}
	\item A double layer GEM using self-stretching technique has been used. It consists of 3$mm$ drift gap, 1mm transfer gap and 1mm induction gap and the effective area is $30\times30cm^2$with $1\times1cm^2$ readout pads.
	\vspace{-0.2cm}
	\item The chip chosen to readout is a tri-thresholds ASIC called MICROROC (MICRO-mesh gaseous structure Read-Out Chip) and the dynamic range is $2fC\sim500fC$ with the RMS noise 0.35$fC$. The system is designed to be a universal scheme for both ECAL and HCAL.
\end{itemize}
}


%----------------------------------------------------------------------------------------
%	CALIBRATION
%----------------------------------------------------------------------------------------

\headerbox{GEM Detector}{name=GEM,column=0,below=introduction}{
\begin{tikzpicture}
% Draw the GEM Hole Picture and text %
\draw (0em,0em) node(ImageGEMHole)[anchor=west]{\includegraphics[width=.4\textwidth]{figures/GEMHole.jpg}};%
\draw (ImageGEMHole.east) node(TextSelfStretch)[anchor=west]{
\begin{minipage}{0.6\textwidth}
	\begin{itemize}
	\item Cu: t = 5$\mu m$
	\vspace{-0.3cm}
	\item Kapton: T = 50$\mu m$
	\vspace{-0.3cm}
	\item Diameter: d = 60$\mu m$
	\vspace{-0.3cm}
	\item D = 80$\mu m$
	\vspace{-0.3cm}
	\item Pitch: 140$\mu m$
	\end{itemize}
\end{minipage}
};
% Draw a line %
%\draw[thick] (ImageGEMHole.south west) ++(0em,-1em) -- ++(20.5em,0em);
% Draw the GEM Structure %
%\draw (ImageGEMHole.south) +(0em,-1.5em) node(ImageStructureOfGEM)[anchor=north]{\includegraphics[width=0.4\textwidth]{figures/StructureOfGEM.jpg}};

% Draw a line %
%\draw[thick] (ImageStructureOfGEM.south west) ++(0em,-1em) -- ++(20.5em,0em);
% Draw the SelfStretching %
\draw (ImageGEMHole.south) +(0em,-3.5em) node(ImageSelfStretch)[anchor=north]{\includegraphics[width=0.4\textwidth]{figures/SelfStreching.png}};
\draw (ImageSelfStretch.east) node(TextSelfStretch)[anchor=west]{
\begin{minipage}{0.6\textwidth}
\begin{itemize}
\item Assembling process is easy and fast
\vspace{-0.3cm}
\item No dead area inside the active area
\vspace{-0.3cm}
\item Uniform gas flow
\vspace{-0.3cm}
\item Detachable
\end{itemize}
\end{minipage}
};
%\draw[thick] (ImageSelfStretch.south west) ++(0em,-2.5em) -- ++(20.5em,0em);
\draw (ImageSelfStretch.south) +(0em,-3.5em) node(ImageGEMDetector)[anchor=north]{\includegraphics[width=0.4\textwidth]{figures/GEMDetector.jpg}};
\draw (ImageGEMDetector.east) node(TextGEMDetector)[anchor=west]{
\begin{minipage}{0.6\textwidth}
\begin{flushleft}
\begin{itemize}
\item Effective zone: $30\times30cm^2$
\vspace{-0.3cm}
\item Drift gap: 3$mm$
\vspace{-0.3cm}
\item Transfer gap: 1$mm$
\vspace{-0.3cm}
\item Induction gap: 1$mm$
\end{itemize}
\end{flushleft}
\end{minipage}
};
\end{tikzpicture}
\vspace{-0.5cm}
}
\headerbox{MICROROC ASIC}{name=Microroc,column=0,below=GEM}{
MICROROC is a 64-channel Semi-Digital readout chip, developed at IN2P3 by OMEGA/LAL. The package of MICROROC chip is TQFP which means the thickness is about 1.4$mm$.\\
%The flowing picture shows the MICROROC chip.
%\begin{center}
%\includegraphics[scale=0.08]{figures/MicrorocChip.jpg}
%\end{center}
Each channel of the MICROROC chip has:
\begin{itemize}
	\item A very low noise charge preamplifier, able to handle a dynamic range from $1fC$ to $500fC$.
	\vspace{-0.2cm}
	\item Two different adjustable shaper. A high gain shaper for small signal and a low gain shaper for large signal.
	\vspace{-0.2cm}
	\item Three comparators for tri-thresholds readout.
	\vspace{-0.2cm}
	\item A random access memory used as a digital buffer.
\end{itemize}
The structure of the analog part is shown below:
\begin{center}
\includegraphics[width=0.8\textwidth]{figures/MicrorocStructureTransparent.png}
\end{center}
The peaking time of the shaper can be adjusted via slow control parameters and the output of two shapers can be selected to a pin for test.
}

%----------------------------------------------------------------------------------------
%	OTHER INSTRUMENTATION
%----------------------------------------------------------------------------------------
\headerbox{Readout System Structure}{name=ReadoutSystemStructure,span=2,column=1,row=1, below=introduction}{ % To reduce this block to 1 column width, remove 'span=2'
The readout system is developed on SRS(Scalable Readout System), which means users can reuse the same system just changing the front-end board.The whole system includes following parts:
\vspace{-0.3cm}
\begin{itemize}
	\item FEB(Front-End Board): Combination of detector and readout ASIC.
	\vspace{-0.3cm}
	\item DIF(Detector InterFace): Control the ASIC and read back data.
	\vspace{-0.3cm}
	\item DAQ: Distribute clock and command. Gather data from DIF.
\end{itemize}
\vspace{-0.3cm}
The structure of the whole system is shown below:
\vspace{-0.3cm}
\begin{center}
\includegraphics[width=0.6\textwidth]{figures/ReadoutStructureTransparent.png}
\end{center}

}


%----------------------------------------------------------------------------------------
%	MIXER vs. SAMPLERS
%----------------------------------------------------------------------------------------
\headerbox{Phase I Design and Test}{name=FirstStageDesignAndTest,span=2,column=1,row=1, below=ReadoutSystemStructure}{
A "Phase I" design is completed to verify the readout structure and test the performance of MICROROC. The front-end ASIC is separated from the detector plane. The system shown below contains the GEM detector, FEB and DIF.\\
\vspace{-0.5cm}
\begin{center}
\includegraphics[height=2.7cm]{figures/PhaseIDesignTransparent.png}
\end{center}
We have a full test of the whole system, including pedestal, calibration, uniform of the GEM and Cosmic-Ray test. The following pictures show the result.\\
\vspace{-0.5cm}
\begin{center}
\includegraphics[height=5.5cm]{figures/TestResultTransparent.png}
\end{center}

}

%----------------------------------------------------------------------------------------
%	MEASUREMENT SETUP
%----------------------------------------------------------------------------------------
\headerbox{Next Step}{name=NextStep,span=2,column=1,below=FirstStageDesignAndTest}{ 
 We project to mount the MICROROC chip on the back side of the readout plane, utilizing blind buried via technology. This action would reduce the dead area of the detector. A 10-layer PCB with 3 ground plane and 2 power plane will guarantee good signal integrity and low crosstalk.  
}


%----------------------------------------------------------------------------------------
%	CONCLUSION
%----------------------------------------------------------------------------------------
\headerbox{Conclusion}{name=conclusion,column=1,below=NextStep,span=2}{
\begin{itemize} 
\item The readout structure is effective and it's able to be used in future design. 
\vspace{-0.2cm}
\item MICROROC can work well with the detector and it's a good choose to readout large area detector.
\vspace{-0.2cm}
\item Include the readout plane and readout ASIC, the FEB can be made within 3$mm$.
\end{itemize}
}


%----------------------------------------------------------------------------------------
%	REFERENCES
%----------------------------------------------------------------------------------------

%\headerbox{References}{name=references,column=2,below=application}{

%\smaller % Reduce the font size in this block
%\renewcommand{\section}[2]{\vskip 0.05em} % Get rid of the default "References" section title
%\nocite{*} % Insert publications even if they are not cited in the poster

%\bibliographystyle{unsrt}
%\bibliographystyle{IEEEtran}
%\bibliography{biblio} % Use biblio.bib as the bibliography file
%}


%----------------------------------------------------------------------------------------
%	ACKNOWLEDGEMENTS
%----------------------------------------------------------------------------------------

\headerbox{Acknowledgements}{name=acknowledgements,column=0,below=conclusion,above=bottom,span=3}{
\smaller 
This study was supported by National Key Programme for S\&T Research and Development (Grant NO.: 2016YFA0400400) and National Science Natural Science Foundation of China (Grant No.11635007). Thanks to St\'ephane CALLIER and Christophe de LA TAILLE at OMEGA - IN2P3/CNRS for helpful discussions and useful suggestions.
} 

\end{poster}

\end{document}